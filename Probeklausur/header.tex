\documentclass[
  bibliography=totoc,     % Literatur im Inhaltsverzeichnis
  captions=tableheading,  % Tabellenüberschriften
  titlepage=firstiscover, % Titelseite ist Deckblatt
]{scrartcl}

% Paket float verbessern
\usepackage{scrhack}

% Warnung, falls nochmal kompiliert werden muss
\usepackage[aux]{rerunfilecheck}
\usepackage{color} 
% unverzichtbare Mathe-Befehle
\usepackage{amsmath}
% viele Mathe-Symbole
\usepackage{amssymb}
% Erweiterungen für amsmath
\usepackage{mathtools}

% Fonteinstellungen
\usepackage{fontspec}
% Latin Modern Fonts werden automatisch geladen
% Alternativ zum Beispiel:
%\setromanfont{Libertinus Serif}
%\setsansfont{Libertinus Sans}
%\setmonofont{Libertinus Mono}

% Wenn man andere Schriftarten gesetzt hat,
% sollte man das Seiten-Layout neu berechnen lassen
\recalctypearea{}

% deutsche Spracheinstellungen
\usepackage{polyglossia}
\setmainlanguage{german}

\usepackage{tikz}
\usepackage[
  math-style=ISO,    % ┐
  bold-style=ISO,    % │
  sans-style=italic, % │ ISO-Standard folgen
  nabla=upright,     % │
  partial=upright,   % ┘
  warnings-off={           % ┐
    mathtools-colon,       % │ unnötige Warnungen ausschalten
    mathtools-overbracket, % │
  },                       % ┘
]{unicode-math}

% traditionelle Fonts für Mathematik
\setmathfont{Latin Modern Math}
% Alternativ zum Beispiel:
%\setmathfont{Libertinus Math}

\setmathfont{XITS Math}[range={scr, bfscr}]
\setmathfont{XITS Math}[range={cal, bfcal}, StylisticSet=1]

% Zahlen und Einheiten
\usepackage[
  locale=DE,                   % deutsche Einstellungen
  separate-uncertainty=true,   % immer Fehler mit \pm
  per-mode=symbol-or-fraction, % / in inline math, fraction in display math
]{siunitx}

% chemische Formeln
\usepackage[
  version=4,
  math-greek=default, % ┐ mit unicode-math zusammenarbeiten
  text-greek=default, % ┘
]{mhchem}

% richtige Anführungszeichen
\usepackage[autostyle]{csquotes}

% schöne Brüche im Text
\usepackage{xfrac}

% Standardplatzierung für Floats einstellen
% allow figures to be placed in the running text by default:
\usepackage{scrhack}
\usepackage{float}
\floatplacement{figure}{htbp}
\floatplacement{table}{htbp}

% Floats innerhalb einer Section halten
\usepackage[
  section, % Floats innerhalb der Section halten
  below,   % unterhalb der Section aber auf der selben Seite ist ok
]{placeins}

% Seite drehen für breite Tabellen: landscape Umgebung
\usepackage{pdflscape}

% Captions schöner machen.
\usepackage[
  labelfont=bf,        % Tabelle x: Abbildung y: ist jetzt fett
  font=small,          % Schrift etwas kleiner als Dokument
  width=0.9\textwidth, % maximale Breite einer Caption schmaler
]{caption}
% subfigure, subtable, subref
\usepackage{subcaption}

% Grafiken können eingebunden werden
\usepackage{graphicx}
% größere Variation von Dateinamen möglich
%\usepackage{grffile}

% schöne Tabellen
\usepackage{booktabs}

% Verbesserungen am Schriftbild
\usepackage{microtype}

% Literaturverzeichnis
\usepackage[
  backend=biber,
]{biblatex}
% Quellendatenbank
\addbibresource{lit.bib}
\addbibresource{programme.bib}

% Hyperlinks im Dokument
\usepackage[
  unicode,        % Unicode in PDF-Attributen erlauben
  pdfusetitle,    % Titel, Autoren und Datum als PDF-Attribute
  pdfcreator={},  % ┐ PDF-Attribute säubern
  pdfproducer={}, % ┘
]{hyperref}
% erweiterte Bookmarks im PDF
\usepackage{bookmark}

% Trennung von Wörtern mit Strichen
\usepackage[shortcuts]{extdash}
\usepackage[utf8]{inputenc}
\usepackage{fancyhdr}
 
\long\def\Rule{\par\par\vspace{4mm}\hrule\vspace{4mm}\par}

\newcommand{\setC}{\mathbbm{C}}
\newcommand{\setK}{\mathbbm{K}}
\newcommand{\setN}{\mathbbm{N}}
\newcommand{\setQ}{\mathbbm{Q}}
\newcommand{\setR}{\mathbbm{R}}
\newcommand{\setNpos}{\mathbbm{N}\setminus\{0\}}
\newcommand{\setZ}{\mathbbm{Z}}
\newcommand{\setZpos}{\mathbbm{Z}\setminus\{0\}}

\setlength{\parindent}{0em}
\topmargin -1.5cm
\oddsidemargin -0.5cm
\evensidemargin -0.5cm
\setlength{\textheight}{9.2in}
\setlength{\textwidth}{6.5in}

\pagenumbering{arabic}

\pagestyle{empty}
\newcounter{aufgabenzaehler}
\newcommand{\topic}[1]{\par\noindent\textbf{{\large #1:}}}
\newcommand{\aufgabe}[1]{\topic{\stepcounter{aufgabenzaehler}\arabic{aufgabenzaehler}.
    Aufgabe}\hfill\textbf{#1 Punkte}\\}
\newcommand{\TutAufgabe}{\topic{\stepcounter{aufgabenzaehler}\arabic{aufgabenzaehler}. Tutoriums-Aufgabe}\\}
\newcommand{\Tip}[1]{\hspace*{\fill}\emph{(Tip: #1)}}
\newcommand{\Quelle}[1]{\hspace*{\fill}\emph{(Quelle: #1)}}
\newcommand{\Loesung}[1]{\hspace*{\fill}\emph{(L\"osung: #1)}}
\setlength{\parindent}{0mm}
\def\Punkte#1{\hbox{\framebox[15mm]{{\large
  \mathstrut#1}}\framebox[15mm]{{\large\mathstrut}}}}
\def\Box#1{\framebox{\phantom{X}}~#1}


\newcommand{\karos}[2]{
  \begin{tikzpicture}
    \draw[step=0.5cm,color=gray!100] (0,0) grid (#1 ,#2cm);
  \end{tikzpicture}
}

%\addtolength{\textheight}{2cm}
\begin{document}
\pagestyle{fancy}
\noindent\textbf{Fachhochschule Südwestfalen}{\small\hfill Probeklausur}\\
Fachbereich I\&W: Deep Learning \\
M.Sc.~ Felix Neubürger\\
\vspace*{-5mm}
\begin{center}\Large
\textbf{Deep Learning}\\
%\textbf{``Diskrete und strukturelle\\ Mathematik f"ur Informatiker''}
\end{center}%\vfill

%\vspace{0.5cm}

\Rule
%\vspace*{6mm}
{
{\begin{minipage}{1\textwidth}\large
\vspace{4mm}
{Name}: \dotfill\\[6mm]
{Vorname}: \dotfill \\[6mm]
{Matr.-Nr.}: \dotfill \\[6mm]
{Prüfungsordnung}:  \dotfill \\[6mm]
  \begin{center}
Mit Ihrer Unterschrift bestätigen Sie, dass Sie gesundheitlich dazu in der Lage sind an der Klausur teilzunehmen.\\[6mm]
\end{center}

{Datum \& Unterschrift}: \dotfill \\[6mm]

%{FB:}:
%\Box{FB 13}\quad\Box{Sonstige:\dotfill}
\end{minipage}}
\vspace{-8mm}
}
\Rule
\begin{center}
\textbf{Erlaubte Hilfsmittel}: Taschenrechner
\end{center}

Mit \textbf{Bleistift} oder \textbf{in rot} geschriebene Aufgaben werden nicht gewertet.
Die Bearbeitungszeit für die Klausur beträgt $90$ Minuten.
Schreiben Sie bitte auf jedes Blatt Ihren \textbf{Namen} und Ihre \textbf{Matrikelnummer} in die dafür vorgesehene Zeile.
\\
\Rule
\bigskip

\begin{center}
Maximal erreichbare Punktzahl: $90$.% ($70$ Punkte entsprechen $100\,\%$).\\
%Zum Bestehen der Klausur sind $40\,\%$ der Gesamtpunktzahl sicherlich ausreichend.\newline


\bigskip
\begin{tabular}{|c|c|c|c|c|c|c|c|c|c|c|c|c|c|}
\hline
Aufgabe &  1 & 2 & 3 & 4 & 5 & 6 & 7 & 8 & 9 & 10 & 11 & 12 & $\,\,\sum$\,\, \\
\hline
Erreichte Punkte &  &  &  &  &  &  &  &  &  &  &  &  & \\
\hline
Erreichbare Punkte & 8 & 7 & 8 & 9 & 9 & 7 & 8 & 8 & 7 & 7 & 8 & 8 & 90 \\
\hline
\end{tabular}
\end{center}

\newpage


\pagestyle{fancy}
\fancyhead[L]{Deep Learning} %Kopfzeile links
\fancyhead[R]{Name: ..............................\\Matr.-Nr.: .......................}

\section*{Aufgabe 1 -- Lineare Algebra -- 8 Punkte}

\textbf{(8 Punkte)}

Gegeben seien die Matrizen:
\begin{equation}
\mathbf{A} = \begin{pmatrix} 2 & 1 \\ 0 & 3 \end{pmatrix}, \quad
\mathbf{B} = \begin{pmatrix} 1 & -1 \\ 2 & 0 \end{pmatrix}
\end{equation}

\begin{itemize}
\item [a)] Berechnen Sie $\mathbf{A} \cdot \mathbf{B}$.
\end{itemize}
\karos{\textwidth}{3}

\begin{itemize}
\item [b)] Berechnen Sie $\mathbf{A}^T$.
\end{itemize}
\karos{\textwidth}{2}

\begin{itemize}
\item [c)] Beschreiben Sie, wie eine $2 \times 2$ Gewichtsmatrix zwei Eingaben mit zwei Neuronen verbindet.
\end{itemize}
\karos{\textwidth}{3}

\newpage

\section*{Aufgabe 2 -- Aktivierungsfunktionen -- 7 Punkte}

\begin{itemize}
\item [a)] Berechnen Sie die Ableitung der Sigmoid-Funktion $\sigma(x) = \frac{1}{1 + e^{-x}}$.
\end{itemize}
\karos{\textwidth}{3}

\begin{itemize}
\item [b)] Berechnen Sie $\sigma(0)$.
\end{itemize}
\karos{\textwidth}{2}

\begin{itemize}
\item [c)] Geben Sie die Ableitung der ReLU-Funktion $\text{ReLU}(x) = \max(0, x)$ an.
\end{itemize}
\karos{\textwidth}{3}

\newpage

\section*{Aufgabe 3 -- Das Perceptron und das XOR-Problem -- 8 Punkte}

\begin{itemize}
\item [a)] Zeichnen Sie die Architektur eines Single-Layer-Perceptrons mit 2 Eingängen und 1 Ausgang.
\end{itemize}
\karos{\textwidth}{2}

\begin{itemize}
\item [b)] Warum kann ein Single-Layer-Perceptron die XOR-Funktion nicht lösen?
\end{itemize}
\karos{\textwidth}{2}

\begin{itemize}
\item [c)] Zeichnen Sie ein Multi-Layer-Perceptron mit 2 Eingängen, 1 Hidden Layer mit 2 Neuronen und 1 Ausgang. Wie viele Gewichte?
\end{itemize}
\karos{\textwidth}{3}

\begin{itemize}
\item [d)] Warum kann dieses Netzwerk XOR lösen?
\end{itemize}
\karos{\textwidth}{3}

\newpage

\section*{Aufgabe 4 -- Partielle Ableitungen und Kettenregel -- 9 Punkte}

Gegeben: $L = \frac{1}{2}(y - a)^2$ mit $y = 1$, $a = \sigma(z)$, $z = w \cdot x + b$.

\begin{itemize}
\item [a)] Berechnen Sie $\frac{\partial L}{\partial a}$.
\end{itemize}
\karos{\textwidth}{2}

\begin{itemize}
\item [b)] Berechnen Sie $\frac{\partial a}{\partial z}$ für Sigmoid $\sigma(z)$.
\end{itemize}
\karos{\textwidth}{2}

\begin{itemize}
\item [c)] Berechnen Sie $\frac{\partial z}{\partial w}$.
\end{itemize}
\karos{\textwidth}{2}

\begin{itemize}
\item [d)] Nutzen Sie die Kettenregel: $\frac{\partial L}{\partial w} = \frac{\partial L}{\partial a} \cdot \frac{\partial a}{\partial z} \cdot \frac{\partial z}{\partial w}$.
\end{itemize}
\karos{\textwidth}{3}

\begin{itemize}
\item [e)] Für $x = 1, w = 0.5, b = 0$: Berechnen Sie $z$, dann $\frac{\partial L}{\partial w}$ (mit $\sigma(0.5) \approx 0.62$).
\end{itemize}
\karos{\textwidth}{3}

\newpage

\section*{Aufgabe 5 -- Forward Pass eines 2-Schicht-Netzwerks -- 9 Punkte}

Gegeben: $\mathbf{x} = \begin{pmatrix} 1 \\ 1 \end{pmatrix}$, $\mathbf{W}^H = \begin{pmatrix} 0.5 & 0.3 \\ 0.2 & 0.4 \end{pmatrix}$, $\mathbf{b}^H = \begin{pmatrix} 0 \\ 0 \end{pmatrix}$

Output Layer: $\mathbf{w}^O = \begin{pmatrix} 1 \\ 1 \end{pmatrix}$, $b^O = 0$

Hidden Layer: ReLU, Output Layer: Linear (keine Aktivierung).

\begin{itemize}
\item [a)] Berechnen Sie $\mathbf{z}^H = \mathbf{W}^H \mathbf{x} + \mathbf{b}^H$.
\end{itemize}
\karos{\textwidth}{2}

\begin{itemize}
\item [b)] Berechnen Sie $\mathbf{a}^H = \text{ReLU}(\mathbf{z}^H)$.
\end{itemize}
\karos{\textwidth}{2}

\begin{itemize}
\item [c)] Berechnen Sie $\hat{y} = \mathbf{w}^O \cdot \mathbf{a}^H + b^O$.
\end{itemize}
\karos{\textwidth}{2}

\begin{itemize}
\item [d)] Mit Label $y = 2$: Berechnen Sie den Verlust $L = \frac{1}{2}(y - \hat{y})^2$.
\end{itemize}
\karos{\textwidth}{2}

\begin{itemize}
\item [e)] Berechnen Sie $\frac{\partial L}{\partial \hat{y}}$.
\end{itemize}
\karos{\textwidth}{2}

\begin{itemize}
\item [f)] Berechnen Sie $\frac{\partial L}{\partial w_1^O}$ mittels Kettenregel.
\end{itemize}
\karos{\textwidth}{2}

\newpage

\section*{Aufgabe 6 -- Gradient Descent -- 7 Punkte}

\begin{itemize}
\item [a)] Schreiben Sie die Gewichtsaktualisierungsformel für Gradient Descent auf.
\end{itemize}
\karos{\textwidth}{2}

\begin{itemize}
\item [b)] Ein Gewicht $w = 0.5$ hat Gradienten $\frac{\partial L}{\partial w} = 0.4$. Aktualisieren Sie mit $\eta = 0.1$.
\end{itemize}
\karos{\textwidth}{2}

\begin{itemize}
\item [c)] Nennen Sie zwei Probleme bei zu hoher Learning Rate.
\end{itemize}
\karos{\textwidth}{2}

\begin{itemize}
\item [d)] Was ist der Unterschied zwischen Batch GD und SGD?
\end{itemize}
\karos{\textwidth}{2}

\newpage

\section*{Aufgabe 7 -- Convolutional Neural Networks (CNNs) -- 8 Punkte}

\begin{itemize}
\item [a)] Nennen Sie zwei Gründe, warum CNNs besser für Bilder sind als vollverbundene Netze.
\end{itemize}
\karos{\textwidth}{2}

\begin{itemize}
\item [b)] Was berechnet die Convolution-Operation?
\end{itemize}
\karos{\textwidth}{2}

\begin{itemize}
\item [c)] Zeichnen Sie ein Max-Pooling Beispiel: $2 \times 2$ Filter über einer $4 \times 4$ Feature Map.
\end{itemize}
\karos{\textwidth}{3}

\begin{itemize}
\item [d)] Nennen Sie zwei CNN-Architekturen und je eine Besonderheit.
\end{itemize}
\karos{\textwidth}{3}

\newpage

\section*{Aufgabe 8 -- RNNs und LSTMs -- 8 Punkte}

\begin{itemize}
\item [a)] Nennen Sie zwei Gründe, warum RNNs für Sequenzen besser sind als vollverbundene Netze.
\end{itemize}
\karos{\textwidth}{2}

\begin{itemize}
\item [b)] Was ist das Vanishing Gradient Problem bei RNNs?
\end{itemize}
\karos{\textwidth}{2}

\begin{itemize}
\item [c)] Skizzieren Sie eine LSTM-Zelle mit vier Komponenten.
\end{itemize}
\karos{\textwidth}{3}

\begin{itemize}
\item [d)] Warum lösen LSTMs das Vanishing Gradient Problem besser?
\end{itemize}
\karos{\textwidth}{3}

\newpage

\section*{Aufgabe 9 -- Generative Modelle -- 7 Punkte}

\begin{itemize}
\item [a)] Unterschied zwischen diskriminativen und generativen Modellen?
\end{itemize}
\karos{\textwidth}{2}

\begin{itemize}
\item [b)] Skizzieren Sie die Architektur eines GAN (Generator und Discriminator).
\end{itemize}
\karos{\textwidth}{3}

\begin{itemize}
\item [c)] Nennen Sie zwei Anwendungen von GANs.
\end{itemize}
\karos{\textwidth}{3}

\newpage

\section*{Aufgabe 10 -- Overfitting und Regularisierung -- 7 Punkte}

\begin{itemize}
\item [a)] Unterschied zwischen Underfitting, Good Fit und Overfitting (mit Skizze)?
\end{itemize}
\karos{\textwidth}{2}

\begin{itemize}
\item [b)] Nennen Sie drei Techniken zur Vermeidung von Overfitting.
\end{itemize}
\karos{\textwidth}{2}

\begin{itemize}
\item [c)] Was ist Dropout und wie funktioniert es beim Training?
\end{itemize}
\karos{\textwidth}{2}

\begin{itemize}
\item [d)] Wie ändert sich Dropout bei der Inferenz?
\end{itemize}
\karos{\textwidth}{2}

\newpage

\section*{Aufgabe 11 -- Gradienten in Vektorform -- 8 Punkte}

Gegeben: $L = \frac{1}{2}(y - \hat{y})^2$ mit $\hat{y} = \mathbf{w}^T \mathbf{a} + b$, $\mathbf{a} = \text{ReLU}(\mathbf{W}\mathbf{x} + \mathbf{b})$

\begin{itemize}
\item [a)] Berechnen Sie $\frac{\partial L}{\partial \hat{y}}$.
\end{itemize}
\karos{\textwidth}{2}

\begin{itemize}
\item [b)] Berechnen Sie $\frac{\partial \hat{y}}{\partial \mathbf{w}}$ (als Vektor).
\end{itemize}
\karos{\textwidth}{2}

\begin{itemize}
\item [c)] Nutzen Sie die Kettenregel: $\frac{\partial L}{\partial \mathbf{w}} = \frac{\partial L}{\partial \hat{y}} \cdot \frac{\partial \hat{y}}{\partial \mathbf{w}}$.
\end{itemize}
\karos{\textwidth}{2}

\begin{itemize}
\item [d)] Gegeben $\mathbf{a} = \begin{pmatrix} 1 \\ 0.5 \end{pmatrix}$, $y = 2$, $\hat{y} = 0.75$: Berechnen Sie $\frac{\partial L}{\partial \mathbf{w}}$ numerisch.
\end{itemize}
\karos{\textwidth}{2}

\begin{itemize}
\item [e)] Warum normalisiert man den Gradienten bei Gradient Descent in die entgegengesetzte Richtung?
\end{itemize}
\karos{\textwidth}{2}

\newpage

\section*{Aufgabe 12 -- Momentum und adaptive Learning Rates -- 8 Punkte}

\begin{itemize}
\item [a)] Erklären Sie die Momentum-Methode kurz.
\end{itemize}
\karos{\textwidth}{2}

\begin{itemize}
\item [b)] Was ist der Unterschied zwischen Momentum und SGD?
\end{itemize}
\karos{\textwidth}{2}

\begin{itemize}
\item [c)] Was ist ADAM und welche Parameter hat es?
\end{itemize}
\karos{\textwidth}{2}

\begin{itemize}
\item [d)] Wann ist ADAM besser als Standard Gradient Descent?
\end{itemize}
\karos{\textwidth}{2}

\begin{itemize}
\item [e)] Was ist der Vorteil von adaptiven Learning Rates?
\end{itemize}
\karos{\textwidth}{2}



\end{document}

